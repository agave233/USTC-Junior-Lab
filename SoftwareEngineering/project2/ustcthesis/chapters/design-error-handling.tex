\chapter{出错处理设计}
\section{数据库出错处理}
多重备份时,应采取何种策略,先利用哪一份备份;系统是否暂停服务等。

当保存用户信息的数据库出错时,暂停整个用户系统的服务,涉及到用户信息的所有操作向客户端返回报错响。

当航班/火车票等信息数据库发生错误时,暂停整个系统的服务,对用户发送的所有查询,预订,更新请求返回报错响应。

当广告活动数据库发生错误时,对于一切推送给用户的信息返回报错反馈.

当预订信息数据库发生错误时,对于一切更新预订的请求返回报错响应,此时客户端应停止运行并且在web端首页提示错误信息,等待后台修复.

在数据库发生错误时,首先尝试使用日志恢复到正常状态;如果不行,尝试恢复本地保存的备份数据;如果本地备份数据也已损坏,尝试恢复容灾节点的备份数据.如果以上操作均失败,则服务器端自行恢复数据库已变的不可能,向客户发出服务已失效的公告.对于用户信息数据库发生的错误,考虑以下解决方案:
在已经告知用户服务器端数据库已毁坏的情况下,由客户端向服务器端发送保存在本地的相关数据,以最大程度的恢复数据.

\section{模块失效处理}

是否整个系统暂停服务,还是维持最小服务状态、如何尽快恢复服务还是删库跑路等。

\subsection{酒店/航班/火车票/景点模块}

当提供相关展示信息的酒店等模块失效时,暂停用户的查询请求和预订请求,其他模块正常工作,当确定模块失效是广泛存在时,系统检测后交给维护人员进行尽快修复,并且尽量由客服人员在客服模块对所有用户说明相关的情况。

\subsection{登录/注册模块}

当登录/注册模块失效时,暂停除首页广告推送之外的所有系统服务,此时用户无法登录系统或者注册新用户,在web端的首页发出维护公告,只能等待开发维护人员修复系统之后才能正常运转。

\subsection{优惠券模块}

优惠券模块失效时,暂停用户的预订请求,因为此时优惠券失效可能导致用户预订时无法正常使用优惠券,其他模块正常工作。首先由本地确定失效原因,如果确定失效原因由代码内部的错误引起,则不断尝试向服务器发送错误信息。当确定模块失效是广泛存在时。由开发者对bug进行修复。

\subsection{活动推送模块}
活动推送模块失效时,暂停web端首页的推送服务,同时暂停服务器端的推荐系统模块,由开发人员检测是web端的代码出现问题还是后台的推荐系统算法存在bug导致,查明问题后尽快修复并且推送给用户。


\subsection{客服模块}

当客服模块失效时,暂停客服模块的工作,其他模块正常工作,认定非网络通信的原因之后,由后台维护人员尽快检测bug并进行修复。


\subsection{后台数据库维护模块}

当后台数据库维护模块失效时,实行上一节介绍的数据库出错处理操作,恢复数据库数据.如果后台数据库维护模块的失效不是由数据错误引起的,则尽快对代码bug进行修复。

后台数据库维护模块失效时,一切发往服务器端的请求都将收到一个保存响应,同时暂停服务器端上其他模块的工作。

\subsection{后台逻辑控制模块}

后台逻辑控制模块模块失效时,暂停所有接收web端请求的工作,尽快对代码bug进行修复。

\subsection{推荐系统模块}

当推荐系统模块失效时,确定是否来自于数据库的问题,并且检测推荐算法,尽快对代码bug进行修复。