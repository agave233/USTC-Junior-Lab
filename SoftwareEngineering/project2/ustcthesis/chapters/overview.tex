\chapter{任务概述}
本系统的目标是实现一个旅行服务管理系统,包括web客户端(包含管理员)、服务器端两个部分。

基于web的客户端面向旅行用户,为用户提供注册登录,机票、火车票、酒店以及门票预订服务,优惠券、活动推送服务以及客服解答服务。

\section{目标}
实现旅行服务管理系统,实现需求规格说明书中所描述的新用户注册和用户登录功能,支持在线预订门票、酒店、火车票和机票功能,用户评价及评分功能,优惠券和活动推荐功能以及客服功能,并且保证系统的健壮性和数据安全。
% ,同时还增加了以管理员身份登录旅行系统,方便可视化的在线管理分析用户,发布编辑新的活动或者门票等服务,

\section{开发与运行环境}

\subsection{开发环境的配置}
\begin{table}[htbp]
\centering
\caption{开发环境的配置} \label{tab:development-environment}
\begin{tabular}{|c|c|c|}
    \hline
    类别 & 标准配置 & 最低配置 \\
    \hline
    计算机硬件 & \tabincell{c}{基于x86结构的CPU\\ 主频>=2.4GHz\\ 内存>=8G\\ 硬盘>=200G} & \tabincell{c}{基于x86结构的CPU\\ 主频>=1.6GHz\\ 内存>=512M\\ 硬盘>=2G} \\
    \hline
    计算机软件 & \tabincell{c}{Linux (kernel version>=4.10)\\ GNU gcc (version>=6.3.1) \\ python (version>=2.7) \\ PyCharm2017.3.3 } & \tabincell{c}{Linux (kernel version>=3.10)\\ GNU gcc (version>=5.4) \\ python (version >=2.7) \\ PyCharm2016.3.3 } \\
    \hline
    网络通信 & \tabincell{c}{至少要有一块可用网卡\\ 能运行IP协议栈即可} & \tabincell{c}{至少要有一块可用网卡\\ 能运行IP协议栈即可} \\
    \hline
    其他 & 采用MySQL数据库 & 采用MySQL数据库 \\
    \hline
\end{tabular}
% \note{这里是表的注释}
\end{table}

\subsection{测试环境的配置}
\begin{table}[htbp]
\centering
\caption{测试环境的配置} \label{tab:test-environment}
\begin{tabular}{|c|c|c|}
    \hline
    类别 & 标准配置 & 最低配置 \\
    \hline
    计算机硬件 & \tabincell{c}{基于x86结构的CPU\\ 主频>=2.4GHz\\ 内存>=8G\\ 硬盘>=200G} & \tabincell{c}{基于x86结构的CPU\\ 主频>=1.6GHz\\ 内存>=512M\\ 硬盘>=2G} \\
    \hline
    计算机软件 & \tabincell{c}{Linux (kernel version>=4.10)\\ Python (version>=2.7)} & \tabincell{c}{Linux (kernel version>=3.10)\\ Python (version>=2.7)} \\
    \hline
    网络通信 & \tabincell{c}{至少要有一块可用网卡\\ 能运行IP协议栈即可} & \tabincell{c}{至少要有一块可用网卡\\ 能运行IP协议栈即可} \\
    \hline
    其他 & \tabincell{c}{采用MySQL数据库} & \tabincell{c}{采用MySQL数据库} \\
    \hline

\end{tabular}
% \note{这里是表的注释}
\end{table}

\subsection{运行环境的配置}
\begin{table}[htbp]
\centering
\caption{运行环境的配置} \label{tab:operation-environment}
\begin{tabular}{|c|c|c|}
    \hline
    类别 & 标准配置 & 最低配置 \\
    \hline
    计算机硬件 & \tabincell{c}{基于x86结构的CPU\\ 主频>=2.4GHz\\ 内存>=8G\\ 硬盘>=200G} & \tabincell{c}{基于x86结构的CPU\\ 主频>=1.6GHz\\ 内存>=512M\\ 硬盘>=2G} \\
    \hline
    计算机软件 & \tabincell{c}{Linux (kernel version>=4.10)\\ Chrome 66.0.3359.139} & \tabincell{c}{Linux (kernel version>=3.10)\\ IE 8以上} \\
    \hline
    网络通信 & \tabincell{c}{至少要有一块可用网卡\\ 能运行IP协议栈即可} & \tabincell{c}{至少要有一块可用网卡\\ 能运行IP协议栈即可} \\
    \hline
    其他 & 支持boostrap框架 & 支持boostrap框架 \\
    \hline

\end{tabular}
% \note{这里是表的注释}
\end{table}

\section{需求概述}
在需求分析报告的基础上稍作修改和扩展,功能需求包括:

\subsection{R.INTF.CALC.001 酒店预定}
\subsubsection{介绍}
通过搜索酒店名称得到相关信息,并且支持预订功能。输入酒店不存在将作为无效输入,在软件中将显示酒店不存在这一信息。不存在错误输入
\subsubsection{输入}
		输入来源:用户

		数量:2,一个预订选项按钮/一个搜索字符串

		度量单位:次数

		时间要求:无

		包含精度和容忍度的有效输入范围:1-20个字符

\subsubsection{处理}
	输入数据的有效性检测:字符及字符串长度

	操作的确切次序,包括各事件的时序:用户输入数据->提交->后端获得数据->数据库查询->结果显示

	对异常情况的回应:通信失败则输出查询失败的信息

	用于把系统输入转换到相应输出的任何方法:数据库查询,映射

	对输出数据的有效性检测:存在性以及数据长度

\subsubsection{输出}
		输出目的地:前端

		数量:多个,酒店名称,价格及其他详细信息

		度量单位:价格/RMB

		时序:接收数据->前端整合渲染

		包含精确度和容忍度的有效输出范围:整数,范围:>0

		对非法值的处理:筛选并返回说明

		错误消息:错误提示信息



\subsection{R.INTF.CALC.002 火车票预订}
\subsubsection{介绍}
通过输入目的地以及时间搜索得到相关火车票信息,并且支持预订功能。输入目的地不存在将作为无效输入,在软件中将显示无相关火车票这一信息。不存在错误输入
\subsubsection{输入}
		输入来源:用户

		数量:3,目的地出发地/时间/预订按钮

		度量单位:次数

		时间要求:无

		包含精度和容忍度的有效输入范围:时间用年月日表示,yyyy-mm-dd的格式,均为正整数,地点为1—20个字符的字符串
\subsubsection{处理}
	输入数据的有效性检测:检查时间格式是否正确,字符串长度

	操作的确切次序,包括各事件的时序:用户输入数据->提交->后端获得数据->数据库查询->结果显示

	对异常情况的回应:通信失败则输出查询失败的信息

	用于把系统输入转换到相应输出的任何方法:数据库查询,映射

	对输出数据的有效性检测:存在性以及数据长度
\subsubsection{输出}
		输出目的地:前端

		数量:多个,出发地->目的地/时间/价格/座位数等信息

		度量单位:价格/RMB,个数等

		时序:接收数据->前端整合渲染

		包含精确度和容忍度的有效输出范围:整数,范围:>0

		对非法值的处理:筛选并返回说明

		错误消息:错误提示信息




\subsection{R.INTF.CALC.003 机票预订}
\subsubsection{介绍}
通过输入目的地以及时间搜索得到相关机票信息,并且支持预订功能。输入目的地不存在将作为无效输入,在软件中将显示无相关机票这一信息。不存在错误输入
\subsubsection{输入}
		输入来源:用户

		数量:3,目的地出发地/时间/预订按钮

		度量单位:次数

		时间要求:无

		包含精度和容忍度的有效输入范围:时间用年月日表示,yyyy-mm-dd的格式,均为正整数,地点为1—20个字符的字符串
\subsubsection{处理}
	输入数据的有效性检测:检查时间格式是否正确,字符串长度

	操作的确切次序,包括各事件的时序:用户输入数据->提交->后端获得数据->数据库查询->结果显示

	对异常情况的回应:通信失败则输出查询失败的信息

	用于把系统输入转换到相应输出的任何方法:数据库查询,映射

	对输出数据的有效性检测:存在性以及数据长度
\subsubsection{输出}
		输出目的地:前端

		数量:多个,出发地->目的地/时间/价格/舱位等信息

		度量单位:价格/RMB,个数等

		时序:接收数据->前端整合渲染

		包含精确度和容忍度的有效输出范围:整数,范围:>0

		对非法值的处理:筛选并返回说明

		错误消息:错误提示信息




\subsection{R.INTF.CALC.004 门票预订}
\subsubsection{介绍}
通过输入景点名称可以搜索得到相关门票信息,并且支持预订功能。输入景点不存在将作为无效输入,在软件中将显示景点不存在这一信息。不存在错误输入
\subsubsection{输入}
		输入来源:用户

		数量:2,景点位置字符串/预订按钮点击

		度量单位:次数

		时间要求:无

		包含精度和容忍度的有效输入范围:1-20字符
\subsubsection{处理}
	输入数据的有效性检测:字符串长度

	操作的确切次序,包括各事件的时序:用户输入数据->提交->后端获得数据->数据库查询->结果显示

	对异常情况的回应:通信失败则输出查询失败的信息

	用于把系统输入转换到相应输出的任何方法:数据库查询,映射

	对输出数据的有效性检测:存在性以及数据长度
\subsubsection{输出}
		输出目的地:前端

		数量:多个,景点、价格、门票类型等

		度量单位:价格/RMB

		时序:接收数据->前端整合渲染

		包含精确度和容忍度的有效输出范围:一位浮点数,范围:无

		对非法值的处理:筛选并返回说明

		错误消息:错误提示信息



\subsection{R.INTF.CALC.005 广告推送}
\subsubsection{介绍}
用户使用时,在网页端显示广告,不需要输入信息。管理员可以发布广告推送。
\subsubsection{输入}
		输入来源:管理员

		数量:一条记录,包含时间/位置/结束时间/价格

		度量单位:价格/RMB

		时间要求:应该在当前日期之后

		包含精度和容忍度的有效输入范围:时间格式为yyyy-mm-dd,价格范围:>1
\subsubsection{处理}
	输入数据的有效性检测:字符串长度,日期格式

	操作的确切次序,包括各事件的时序:管理员输入数据->提交->后端获得数据->录入数据库供查询

	对异常情况的回应:通信失败则输出提交失败的信息

	用于把系统输入转换到相应输出的任何方法:数据库查询,映射

	对输出数据的有效性检测:数据库完整性检测
\subsubsection{输出}
		输出目的地:服务器

		数量:多个,与输入一致

		度量单位:与输入一致

		时序:生成数据库语句->存入数据库

		包含精确度和容忍度的有效输出范围:筛选并返回说明

		对非法值的处理:错误提示信息



\subsection{R.INTF.CALC.006 用户评价}
\subsubsection{介绍}
用户通过该系统可以对景点或者酒店等其它内容进行评价,输入为用户的评价,非法条件为超过字数限制,对于非法输入将显示错误信息
\subsubsection{输入}
		输入来源:用户

		数量:2,评分/评价内容

		度量单位:无

		时间要求:在使用预订的一定期限内

		包含精度和容忍度的有效输入范围:每段评价不超过100个字符,评分为0.0-5.0的一位浮点数
\subsubsection{处理}
	输入数据的有效性检测:检测输入信息的字符个数,评分检测

	操作的确切次序,包括各事件的时序:用户输入数据->提交->后端获得数据->录入数据库

	对异常情况的回应:通信失败则输出评价失败的信息

	用于把系统输入转换到相应输出的任何方法:直接输出

	对输出数据的有效性检测:数据库完整性检测
\subsubsection{输出}
		输出目的地:服务器

		数量:2

		度量单位:无

		时序:生成数据库语句->存入数据库

		包含精确度和容忍度的有效输出范围:输出小于100个字符,评分为

		对非法值的处理:反馈错误提示信息,输出长度超过限制的信息等




\subsection{R.INTF.CALC.007 优惠卷}
\subsubsection{介绍}
用户使用时,在网页端显示优惠卷,用户可以领取。管理员可以在系统中发布优惠券。
\subsubsection{输入}
		输入来源:用户/管理员

		数量:1

		度量单位:张

		时间要求:用户在优惠券有效日期内提交领取

		包含精度和容忍度的有效输入范围:提交记录的输入日期按照yyyy-mm-dd格式,折扣为1-99的整型数。
\subsubsection{处理}
	输入数据的有效性检测:检测时间输入合法性,折扣输入合法。

	操作的确切次序,包括各事件的时序:用户/管理员输入数据->提交->后端获得数据->录入数据库

	对异常情况的回应:通信失败则输出领取/发布优惠券失败的信息

	用于把系统输入转换到相应输出的任何方法:直接输出

	对输出数据的有效性检测:数据库完整性检测,用户是否已经领取过该优惠券
\subsubsection{输出}
		输出目的地:服务器

		数量:1

		度量单位:张

		时序:生成数据库语句->存入/更新数据库

		包含精确度和容忍度的有效输出范围:无

		对非法值的处理:反馈错误提示信息




\subsection{R.INTF.CALC.008 活动推送}
\subsubsection{介绍}
在网页端显示活动,不需要输入信息。管理员可以在系统中发布/编辑新活动。
\subsubsection{输入}
		输入来源:管理员

		数量:1

		度量单位:项

		时间要求:在活动截止日期前发布

		包含精度和容忍度的有效输入范围:时间格式,按照yyyy-mm-dd hh:mm:ss,价格范围:>0
\subsubsection{处理}
	输入数据的有效性检测:检测时间输入合法性,价格输入合法。

	操作的确切次序,包括各事件的时序:管理员输入数据->提交->后端获得数据->录入数据库

	对异常情况的回应:通信失败则输出发布/更新新活动失败的信息

	用于把系统输入转换到相应输出的任何方法:直接输出

	对输出数据的有效性检测:数据库完整性检测
\subsubsection{输出}
		输出目的地:服务器

		数量:1

		度量单位:项

		时序:生成数据库语句->存入/更新数据库

		包含精确度和容忍度的有效输出范围:无

		对非法值的处理:反馈错误提示信息



\subsection{R.INTF.CALC.009 客服服务}
\subsubsection{介绍}
通过系统能够与客服进行即时通讯,输入同样用字数限制,对于非法输入将显示错误信息
\subsubsection{输入}
		输入来源:用户/客服

		数量:不限

		度量单位:消息条数

		时间要求:延迟小于0.1s

		包含精度和容忍度的有效输入范围:每段消息不超过100个字符
\subsubsection{处理}
	输入数据的有效性检测:输入信息的字符个数

	操作的确切次序,包括各事件的时序:用户输入数据->提交->后端获得数据->客服回复信息->信息显示(反之亦然)

	对异常情况的回应:通信失败则输出信息未成功发送

	用于把系统输入转换到相应输出的任何方法:无

	对输出数据的有效性检测:检测是否含有非法字符
\subsubsection{输出}
		输出目的地:前端

		数量:不限

		度量单位:消息条数

		时序:延迟小于0.1s

		包含精确度和容忍度的有效输出范围:应该小于显示框的容纳字符数

		对非法值的处理:反馈错误提示信息



\subsection{R.INTF.CALC.010 用户登录}
\subsubsection{介绍}
通过输入用户名与密码登录系统,错误输入为用户不存在或者密码错误,针对相应错误系统会分别提示错误信息
\subsubsection{输入}
		输入来源:用户

		数量:2,用户名和密码

		度量单位:位

		时间要求:无

		包含精度和容忍度的有效输入范围:无
\subsubsection{处理}
	输入数据的有效性检测:密码是否正确

	操作的确切次序,包括各事件的时序:用户输入数据->提交->后端获得数据->数据库查询并且验证->若成功则显示登陆

	对异常情况的回应:通信失败则输出信息未成功提交,查询不匹配则输出相关信息

	用于把系统输入转换到相应输出的任何方法:数据库查询

	对输出数据的有效性检测:无
\subsubsection{输出}
		前端返回登录后的首页



\subsection{R.INTF.CALC.011 注册用户/修改个人信息}
\subsubsection{介绍}
通过输入用户名与密码进行注册,错误输入为用户名已经注册,系统会提示相应的错误信息。用户在登录状态下可以提交信息修改。
\subsubsection{输入}
		输入来源:用户

		数量:多个,用户名和密码及其他信息

		度量单位:位

		时间要求:无

		包含精度和容忍度的有效输入范围:用户名用英文字符表示,密码仅包含数字和字母
\subsubsection{处理}
	输入数据的有效性检测:用户名是否重复

	操作的确切次序,包括各事件的时序:用户输入数据->提交->后端获得数据->数据库查询是否重名->若成功则注册用户即插入信息到数据库

	对异常情况的回应:通信失败则输出信息未成功提交,查询到重名则输出相关信息

	用于把系统输入转换到相应输出的任何方法:数据库查询

	对输出数据的有效性检测:无
\subsubsection{输出}
		输出目的地:服务器

		数量:1

		度量单位:记录

		时序:生成数据库语句->存入/更新数据库

		包含精确度和容忍度的有效输出范围:无

		对非法值的处理:反馈错误信息


\newpage
\section{条件与限制}
\subsection{开发平台与工具}
本项目选用Windows10系统作为主要的开发平台,同时选择win64下的Pycharm作为开发IDE,基于python的flask框架开发,用阿里云的Ubuntu16.04lts作为服务器后台。
\subsection{软件开发生命周期模型}
采用瀑布模型作为软件生命周期模型,因为瀑布模型适用于需求比较固定的情形,并且实行起来较为简单。
\subsection{法律}
本管理系统所提供的所有航班、门票信息均来自于互联网上的正规网站发布,保证真实合法,景点门票和旅行社活动的相关信息来源于国内知名
旅游网站,在管理系统真正应用时会和相关的部门人员取得合作关系,保证门票/航班等代理预订有效且在法律上允许。
\subsection{技术}
本系统主要是基于后端的flask框架进行实现,flask是一个轻量级的web后端框架,结合mysql5.7实现后台数据库管理,在流量不是巨大的情况下保证可行,但由于后台开发技术知识有限和开发时间短,所以难免有bug疏漏和性能上的劣势。同时由于团队中缺少精通前端开发的人员,所以对于web前端采用的是Bootstrap框架,对于前端并没有做深入的美化,因此本项目发布在github上开源以供进一步完善的参考。
\subsection{经费}
开发过程无经费,采用的服务器主机为阿里云的学生服务器,所以带宽和存储等配置相对较差。