\chapter{引言}
\section{编写目的}
在本项目的前一阶段,也就是需求分析阶段,已经将系统用户对本系统的需求做了详细的阐述,这些用户需求已经在上一阶段中对不同用户所提出的不同功能,实现的各种效果做了调研工作,并在需求规格说明书中得到详尽得叙述及阐明。

本阶段已在系统的需求分析的基础上,对即时聊天工具做概要设计。主要解决了实现该系统需求的程序模块设计问题。包括如何把该系统划分成若干个模块、决定各个模块之间的接口、模块之间传递的信息,以及数据结构、模块结构的设计等。在以下的概要设计报告中将对在本阶段中对系统所做的所有概要设计进行详细的说明,在设计过程中起到了提纲挈领的作用。

在下一阶段的详细设计中,程序设计员可参考此概要设计报告,在概要设计即时聊天工具所做的模块结构设计的基础上,对系统进行详细设计。在以后的软件测试以及软件维护阶段也可参考此说明书,以便于了解在概要设计过程中所完成的各模块设计结构,或在修改时找出在本阶段设计的不足或错误。


\section{项目背景}
随着社会的不断发展和人们对于生活质量的要求日益提高,旅游出行几乎成为每个人经常性的活动,而且随着互联网的移动网络的飞速发展,人们越来越热衷于从线上进行预订门票、车票,安排出游活动,这一高效简便的选择大大节约了人们的时间,同时有了更多更好的选择。通过互联网在线选择旅行的各项需求,用户不用去到旅行社就能获得优质的在线旅行服务,世界各地景点的门票、乘坐的航班甚至各个酒店的详细信息、优惠券信息都可以实时展现给用户,线上的旅行服务管理系统能带给用户全方位的选择和优质的旅行服务体验,因此本项目基于这一出发点,从用户日常进行旅游的需求出发,开发一个提供高效服务和方便管理维护的旅行服务管理系统。

\section{术语}
[列出本文档中所用到的专门术语的定义和外文缩写的原词组]
\begin{table}[htbp]
\centering
\caption{术语表} \label{tab:terminology}
\begin{tabular}{|c|c|}
    \hline
    缩写、术语 & 解释 \\
    \hline
    SQL & SQL指结构化查询语言, 使我们有能力访问数据库 \\
    Bootstrap & Bootstrap是基于HTML、CSS、JavaScript的一种前端框架 \\
    gunicorn & 一种wsgi容器,只支持在Unix系统上运行,来源于Ruby的unicorn项目 \\
    \hline
\end{tabular}
% \note{这里是表的注释}
\end{table}